\documentclass{statsmsc}

\title{Title of the Thesis}
\author{FIRSTNAME LASTNAME}
\CID{01234567}
\supervisor{SUPERVISORNAME and COSUPERVISORNAME}
\date{1 May 2022}
%For today's date, use:
%\date{\today}
\logoimg{}


% THIS IS WHERE NEW COMMANDS CAN BE DEFINED
% commands below only used in the proof; otherwise can be deleted
\newcommand{\consta}{a}
\newcommand{\X}{X}
\newcommand{\EE}[1]{ \mathrm{E} [ #1 ] }
\newcommand{\inparenth}[1]{\left( #1 \right)}

\begin{document}

% Generates the Title Page
\maketitle


% Generates plagiarism declaration
\declarationname{STUDENT'S NAME}
\declarationdate{DATE}
\declaration 


\begin{abstract}
    ABSTRACT GOES HERE
\end{abstract}

\begin{acknowledgements}
    ANY ACKNOWLEDGEMENTS GO HERE
\end{acknowledgements}

% add glossary?

% VERY IMPORTANT
% This command switches from Roman to Arabic numbering for main part of thesis
\mainmatter


\chapter{Introduction}

The introduction section goes here\footnote{Tip: write this section last.}.

\chapter{Background}

Background chapter.

\section{My section}

Section content goes here. 

\subsection{My subsection}

A subsection.

\subsubsection{My subsubsection}

A subsubsection.

\section{Figures}

It is better to create figures in a vector-based format, such as PDF.

\begin{figure}[!h]
    \centering
    \includegraphics[width=\textwidth]{figures/fig1.pdf}
    \caption{Remember to make fonts in figures large enough (compare the two figures.)}
    \label{fig:fig1}
\end{figure}

\section{Tables}

\label{sec:tablesection}

Here is an example of a table

\begin{table}[ht]
\centering
\begin{tabular}{rr}
  \hline
    $z$& $\textrm{P}(Z < z)$ \\
  \hline
    1.281& 0.900\\
    1.645& 0.950\\
    1.960& 0.975\\
    2.326& 0.990 \\
    2.576& 0.995 \\
   \hline
\end{tabular}
    \caption{Partial table showing values of $z$ for $\textrm{P}(Z < z)$, 
    where $Z$ has a standard normal distribution.}
    \label{tab:normal}
\end{table}



\section{Referencing sources, sections and items}

A good book on the bootstrap is \cite{efrontib}, although the idea
appeared in an earlier paper \citep{efron1979}.

Note that to make the references appear, you will need to compile
the bibtex, otherwise you may just see question marks where the references
should be.


\subsection{Referencing sections, results and equations}

Theorem~\ref{thm:theorem1} is proved in Section~\ref{sec:defnthms}; see 
Equation~\eqref{eqn:second}.



\subsubsection{Referencing tables and figures}

When labelling figures and tables, it is important that the label command
\texttt{\textbackslash label\{LABELNAME\}} comes \textbf{after} the caption command.
See Table~\ref{tab:normal} and Figure~\ref{fig:fig1} above.

\subsection{Quoting sources}
If you wish to quote a source, be sure to use quotation marks and cite the
reference. The \texttt{\textbackslash{usequote}} command is useful here:

\usequote{It was the best of times, it was the worst of times, it was the age of wisdom, it was the age of foolishness, it was the epoch of belief, it was the epoch of incredulity, it was the season of Light, it was the season of Darkness, it was the spring of hope, it was the winter of despair, we had everything before us, we had nothing before us, we were all going direct to Heaven, we were all going direct the other way - in short, the period was so far like the present period, that some of its noisiest authorities insisted on its being received, for good or for evil, in the superlative degree of comparison only.} \citep{dickens1859}

% forcing a page break
\clearpage


\section{Definitions, theorems and examples}
\label{sec:defnthms}

The following environments are supported:
Definition, Theorem, Proof, Proposition, Lemma, Remark, Example.

\begin{definition}
    The \textbf{variance} of a random variable $X$ is defined as
    \begin{equation}
        \mathrm{Var}(X) = \mathrm{E}[(X - \mathrm{E}[X])^2].
    \end{equation}
\end{definition}

\begin{theorem}
    \label{thm:theorem1}
Given a random variable $X$, over all values $a \in \mathbb{R}$, 
\begin{equation}
    \min_{a \in \mathbb{R}} \mathrm{E}[(X - a)^2] 
    = \mathrm{E}[(X - \mathrm{E}[X])^2].
    \label{eqn:second}
\end{equation}
\end{theorem}

\begin{proof}   
    Starting with the left-hand side,
\begin{align}
% Example of using \newcommands; see above
\EE{ \inparenth{\X - \consta}^2 } 
&=  \EE{  \inparenth{\X - \EE{\X} + \EE{\X} - \consta}^2 } 
    \nonumber \\
&=  \EE{  \inparenth{ \X - \EE{\X} }^2 } 
    + 2 \EE{ \inparenth{ \X - \EE{\X} } \inparenth{ \EE{\X} - \consta} } +  
  \EE{ \inparenth{ \EE{\X} - \consta}^2  }    
    \nonumber \\
&=  \EE{  \inparenth{ \X - \EE{\X} }^2 } +  \inparenth{ \EE{\X} - \consta}^2
    \nonumber \\
& \geq \EE{  \inparenth{ \X - \EE{\X} }^2 },  
\nonumber 
\end{align}
since $\EE{\X}$ is a real number and $\inparenth{ \EE{\X} - \consta}^2 \geq 0$,
and the third line follows from linearity of expectation:
\begin{align}
    \EE{ \inparenth{ \X - \EE{\X} } \inparenth{ \EE{\X} - \consta} }
    =
    \inparenth{ \EE{\X} - \consta} \EE{ \inparenth{ \X - \EE{\X} } }
    =
    \inparenth{ \EE{\X} - \consta}  \inparenth{\EE{ \X} - \EE{\X} } 
    =
    0,
    \nonumber
\end{align}
    since $\EE{\EE{\X}} = \EE{\X}$, which proves the result.
\end{proof}

\begin{remark}
    This theorem shows that that the minimum of the quantity
    $\mathrm{E}[(X - a)^2]$ is equal to $\mathrm{Var}(X)$.
    In some sense, this makes the variance a natural measure of dispersion if 
    we are taking the metric to be the squared deviation of $X$.
\end{remark}


\begin{lemma}[Stein's Lemma]
    Let $X \sim \mathrm{N}(\mu, \sigma^2)$, and let $g$ be a differentiable 
    function satisfying $\mathrm{E}[|g'(X)|] < \infty$. Then
    \begin{equation}
        \mathrm{E}[g(X)(X-\mu)] = \sigma^2 \mathrm{E}[g'(X)].
        \nonumber
    \end{equation}
\end{lemma}

\begin{proposition}[Popoviciu's inequality]
    Suppose that the random variable $X$ is known to only take values in the 
    bounded range $[a, b]$. Then  
    \begin{align}
        \mathrm{Var}[X] \leq \dfrac{(b-a)^2}{4}.
        \nonumber
    \end{align}
\end{proposition}

\begin{example}
    Suppose $X \sim \mathrm{Bern} (p)$, for some $p \in [0,1]$. Then, 
    since $X \in \{0, 1\}$, $X$ is bounded between $0$ and $1$ and so
    $\mathrm{Var}[X] \leq \tfrac{1}{4}$.
\end{example}



\chapter{Conclusion}


Conclusion goes here. 





\clearpage
 %% reset page counter and start appendix pages with A
\pagenumbering{arabic}
\renewcommand*{\thepage}{A\arabic{page}}

%% Appendix goes here
%\appendix
%
%\chapter{Appendix title}
%
%Appendix goes here.


%%References part of appendices
% References: modify the file refs.bib
\bibliographystyle{plainnat}
\bibliography{refs}


\end{document}
